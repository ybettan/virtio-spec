\section{Increment-edu Device}\label{sec:Device Types / Increment-edu Device}

The virtio Increment-edu device is a simple virtual incremental device that
increment the number represented by its input by 1.
Read and write requests are placed in the queue, and serviced
(probably out of order) by the device.
The main purpose of this device is to be used as a reference for new contributors
if they wish to create a new virtio device, therefore, it contains minimum
functionality and mostly describe the building blocks of the virtio-protocol.

\subsection{Device ID}\label{sec:Device Types / Increment-edu Device / Device ID}
  21

\subsection{Virtqueues}\label{sec:Device Types / Increment-edu Device / Virtqueues}
\begin{description}
\item[0] requestq
\end{description}

\subsection{Feature bits}\label{sec:Device Types / Increment-edu Device / Feature bits}

None currently defined.

\subsection{Device configuration layout}\label{sec:Device Types / Increment-edu Device / Device configuration layout}

None currently defined.

\subsection{Device Initialization}\label{sec:Device Types / Increment-edu Device / Device Initialization}

\begin{enumerate}
\item The virtqueue is initialized
\end{enumerate}

\subsection{Device Operation}\label{sec:Device Types / Increment-edu Device / Device Operation}

When the driver need computation, it places a request that consist of both
input and output elements into the queue.
The first element is used as an input for the device and contain a 4 bytes
integer, and the second is given for the device to fill the result in it as a
4 bytes integer indeed.

Both the input and the output are in little-endian bytes order.

The device is increasing this integer by 1, meaning the LSB will be
increased by 1 and if needed the carry will be carried to the next byte.

If the device get a number that is out of the range of a 4 bytes integer only
the lower bytes that fit the size of a 4 bytes integer will represent the input
and the upper bytes will be ignored.
If the result is out of range then only the lower bytes will be written as
result as well.

\drivernormative{\subsubsection}{Device Operation}{Device Types / Increment-edu Device / Device Operation}

The driver MUST place the 2 elements in the same request (buffer) in order to
make a request atomically handled by the device, the first element contains the
input and should be read-only and the second should be write-only.

\devicenormative{\subsubsection}{Device Operation}{Device Types / Increment-edu Device / Device Operation}

The device MUST place the result inside the output element allocated by the
driver.
